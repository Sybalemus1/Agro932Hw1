\PassOptionsToPackage{unicode=true}{hyperref} % options for packages loaded elsewhere
\PassOptionsToPackage{hyphens}{url}
%
\documentclass[]{article}
\usepackage{lmodern}
\usepackage{amssymb,amsmath}
\usepackage{ifxetex,ifluatex}
\usepackage{fixltx2e} % provides \textsubscript
\ifnum 0\ifxetex 1\fi\ifluatex 1\fi=0 % if pdftex
  \usepackage[T1]{fontenc}
  \usepackage[utf8]{inputenc}
  \usepackage{textcomp} % provides euro and other symbols
\else % if luatex or xelatex
  \usepackage{unicode-math}
  \defaultfontfeatures{Ligatures=TeX,Scale=MatchLowercase}
\fi
% use upquote if available, for straight quotes in verbatim environments
\IfFileExists{upquote.sty}{\usepackage{upquote}}{}
% use microtype if available
\IfFileExists{microtype.sty}{%
\usepackage[]{microtype}
\UseMicrotypeSet[protrusion]{basicmath} % disable protrusion for tt fonts
}{}
\IfFileExists{parskip.sty}{%
\usepackage{parskip}
}{% else
\setlength{\parindent}{0pt}
\setlength{\parskip}{6pt plus 2pt minus 1pt}
}
\usepackage{hyperref}
\hypersetup{
            pdftitle={hw1},
            pdfauthor={Santos Barrera},
            pdfborder={0 0 0},
            breaklinks=true}
\urlstyle{same}  % don't use monospace font for urls
\usepackage[margin=1in]{geometry}
\usepackage{color}
\usepackage{fancyvrb}
\newcommand{\VerbBar}{|}
\newcommand{\VERB}{\Verb[commandchars=\\\{\}]}
\DefineVerbatimEnvironment{Highlighting}{Verbatim}{commandchars=\\\{\}}
% Add ',fontsize=\small' for more characters per line
\usepackage{framed}
\definecolor{shadecolor}{RGB}{248,248,248}
\newenvironment{Shaded}{\begin{snugshade}}{\end{snugshade}}
\newcommand{\AlertTok}[1]{\textcolor[rgb]{0.94,0.16,0.16}{#1}}
\newcommand{\AnnotationTok}[1]{\textcolor[rgb]{0.56,0.35,0.01}{\textbf{\textit{#1}}}}
\newcommand{\AttributeTok}[1]{\textcolor[rgb]{0.77,0.63,0.00}{#1}}
\newcommand{\BaseNTok}[1]{\textcolor[rgb]{0.00,0.00,0.81}{#1}}
\newcommand{\BuiltInTok}[1]{#1}
\newcommand{\CharTok}[1]{\textcolor[rgb]{0.31,0.60,0.02}{#1}}
\newcommand{\CommentTok}[1]{\textcolor[rgb]{0.56,0.35,0.01}{\textit{#1}}}
\newcommand{\CommentVarTok}[1]{\textcolor[rgb]{0.56,0.35,0.01}{\textbf{\textit{#1}}}}
\newcommand{\ConstantTok}[1]{\textcolor[rgb]{0.00,0.00,0.00}{#1}}
\newcommand{\ControlFlowTok}[1]{\textcolor[rgb]{0.13,0.29,0.53}{\textbf{#1}}}
\newcommand{\DataTypeTok}[1]{\textcolor[rgb]{0.13,0.29,0.53}{#1}}
\newcommand{\DecValTok}[1]{\textcolor[rgb]{0.00,0.00,0.81}{#1}}
\newcommand{\DocumentationTok}[1]{\textcolor[rgb]{0.56,0.35,0.01}{\textbf{\textit{#1}}}}
\newcommand{\ErrorTok}[1]{\textcolor[rgb]{0.64,0.00,0.00}{\textbf{#1}}}
\newcommand{\ExtensionTok}[1]{#1}
\newcommand{\FloatTok}[1]{\textcolor[rgb]{0.00,0.00,0.81}{#1}}
\newcommand{\FunctionTok}[1]{\textcolor[rgb]{0.00,0.00,0.00}{#1}}
\newcommand{\ImportTok}[1]{#1}
\newcommand{\InformationTok}[1]{\textcolor[rgb]{0.56,0.35,0.01}{\textbf{\textit{#1}}}}
\newcommand{\KeywordTok}[1]{\textcolor[rgb]{0.13,0.29,0.53}{\textbf{#1}}}
\newcommand{\NormalTok}[1]{#1}
\newcommand{\OperatorTok}[1]{\textcolor[rgb]{0.81,0.36,0.00}{\textbf{#1}}}
\newcommand{\OtherTok}[1]{\textcolor[rgb]{0.56,0.35,0.01}{#1}}
\newcommand{\PreprocessorTok}[1]{\textcolor[rgb]{0.56,0.35,0.01}{\textit{#1}}}
\newcommand{\RegionMarkerTok}[1]{#1}
\newcommand{\SpecialCharTok}[1]{\textcolor[rgb]{0.00,0.00,0.00}{#1}}
\newcommand{\SpecialStringTok}[1]{\textcolor[rgb]{0.31,0.60,0.02}{#1}}
\newcommand{\StringTok}[1]{\textcolor[rgb]{0.31,0.60,0.02}{#1}}
\newcommand{\VariableTok}[1]{\textcolor[rgb]{0.00,0.00,0.00}{#1}}
\newcommand{\VerbatimStringTok}[1]{\textcolor[rgb]{0.31,0.60,0.02}{#1}}
\newcommand{\WarningTok}[1]{\textcolor[rgb]{0.56,0.35,0.01}{\textbf{\textit{#1}}}}
\usepackage{longtable,booktabs}
% Fix footnotes in tables (requires footnote package)
\IfFileExists{footnote.sty}{\usepackage{footnote}\makesavenoteenv{longtable}}{}
\usepackage{graphicx,grffile}
\makeatletter
\def\maxwidth{\ifdim\Gin@nat@width>\linewidth\linewidth\else\Gin@nat@width\fi}
\def\maxheight{\ifdim\Gin@nat@height>\textheight\textheight\else\Gin@nat@height\fi}
\makeatother
% Scale images if necessary, so that they will not overflow the page
% margins by default, and it is still possible to overwrite the defaults
% using explicit options in \includegraphics[width, height, ...]{}
\setkeys{Gin}{width=\maxwidth,height=\maxheight,keepaspectratio}
\setlength{\emergencystretch}{3em}  % prevent overfull lines
\providecommand{\tightlist}{%
  \setlength{\itemsep}{0pt}\setlength{\parskip}{0pt}}
\setcounter{secnumdepth}{0}
% Redefines (sub)paragraphs to behave more like sections
\ifx\paragraph\undefined\else
\let\oldparagraph\paragraph
\renewcommand{\paragraph}[1]{\oldparagraph{#1}\mbox{}}
\fi
\ifx\subparagraph\undefined\else
\let\oldsubparagraph\subparagraph
\renewcommand{\subparagraph}[1]{\oldsubparagraph{#1}\mbox{}}
\fi

% set default figure placement to htbp
\makeatletter
\def\fps@figure{htbp}
\makeatother


\title{hw1}
\author{Santos Barrera}
\date{2/26/2020}

\begin{document}
\maketitle

\hypertarget{ngs-data-simulation-using-wgsim}{%
\section{\texorpdfstring{NGS data simulation using
\texttt{wgsim}}{NGS data simulation using wgsim}}\label{ngs-data-simulation-using-wgsim}}

\hypertarget{type-in-the-following-command}{%
\paragraph{Type in the following
command:}\label{type-in-the-following-command}}

\begin{Shaded}
\begin{Highlighting}[]
\ExtensionTok{wgsim}\NormalTok{ lambda.fa -e 0 -d 500 -N 5000 -1 100 -2 100 -r 0.01  -R 0 -X 0 -S 1234567 -h l1.read1.fq l1.read2.fq}
\end{Highlighting}
\end{Shaded}

\begin{center}\rule{0.5\linewidth}{0.5pt}\end{center}

\hypertarget{download-reference-from-ensemblplants}{%
\section{Download Reference from
EnsemblPlants}\label{download-reference-from-ensemblplants}}

Maize \href{https://plants.ensembl.org/Zea_mays/Info/Index}{reference
genome}

\hypertarget{change-to-largedatalab4-folder}{%
\paragraph{\texorpdfstring{Change to
\texttt{largedata\textbackslash{}lab4}
folder:}{Change to largedata\textbackslash{}lab4 folder:}}\label{change-to-largedatalab4-folder}}

\begin{Shaded}
\begin{Highlighting}[]
\BuiltInTok{cd}\NormalTok{ largedata}
\FunctionTok{mkdir}\NormalTok{ Thetas_Fst}
\BuiltInTok{cd}\NormalTok{ Thetas_Fst}
\end{Highlighting}
\end{Shaded}

\hypertarget{then-use-wget-to-download-the-reference-genome}{%
\paragraph{\texorpdfstring{Then use \texttt{wget} to download the
reference
genome:}{Then use wget to download the reference genome:}}\label{then-use-wget-to-download-the-reference-genome}}

\begin{Shaded}
\begin{Highlighting}[]
\FunctionTok{wget}\NormalTok{ ftp://ftp.ensemblgenomes.org/pub/plants/release-46/fasta/zea_mays/dna/Zea_mays.B73_RefGen_v4.dna.chromosome.Mt.fa.gz}

\CommentTok{### then unzip it}
\FunctionTok{gunzip}\NormalTok{ Zea_mays.B73_RefGen_v4.dna.chromosome.Mt.fa.gz}
\end{Highlighting}
\end{Shaded}

\begin{center}\rule{0.5\linewidth}{0.5pt}\end{center}

\hypertarget{ngs-data-simulation-using-wgsim-1}{%
\section{\texorpdfstring{NGS data simulation using
\texttt{wgsim}}{NGS data simulation using wgsim}}\label{ngs-data-simulation-using-wgsim-1}}

\hypertarget{type-in-the-following-command-1}{%
\paragraph{Type in the following
command:}\label{type-in-the-following-command-1}}

\begin{Shaded}
\begin{Highlighting}[]
\ExtensionTok{wgsim}\NormalTok{ Zea_mays.B73_RefGen_v4.dna.chromosome.Mt.fa \textbackslash{}}
\NormalTok{-e 0 -d 500 -N 5000 -1 100 -2 100 -r 0.01  \textbackslash{}}
\NormalTok{-R 0 -X 0 -S 1234567 l1.read1.fq l1.read2.fq}
\end{Highlighting}
\end{Shaded}

\begin{itemize}
\tightlist
\item
  Reference (about 500k)

  \begin{itemize}
  \tightlist
  \item
    \texttt{Zea\_mays.B73\_RefGen\_v4.dna.chromosome.Mt.fa}
  \end{itemize}
\item
  20x coverage

  \begin{itemize}
  \tightlist
  \item
    \texttt{N\ 5000}
  \end{itemize}
\item
  PE 100bp

  \begin{itemize}
  \tightlist
  \item
    \texttt{-1\ 100\ -2\ 100}
  \end{itemize}
\item
  Only SNP no Indel

  \begin{itemize}
  \tightlist
  \item
    \texttt{-R\ 0\ -X\ 0}
  \end{itemize}
\item
  Mutation rate is low

  \begin{itemize}
  \tightlist
  \item
    \texttt{-r\ 0.01}
  \end{itemize}
\end{itemize}

\begin{center}\rule{0.5\linewidth}{0.5pt}\end{center}

\hypertarget{ngs-data-simulation-using-wgsim-2}{%
\section{\texorpdfstring{NGS data simulation using
\texttt{wgsim}}{NGS data simulation using wgsim}}\label{ngs-data-simulation-using-wgsim-2}}

\hypertarget{simulate-10-individals-population-1}{%
\subsection{simulate 10 individals population
1}\label{simulate-10-individals-population-1}}

\begin{Shaded}
\begin{Highlighting}[]
\KeywordTok{for} \ExtensionTok{i}\NormalTok{ in }\DataTypeTok{\{1..10\}}\KeywordTok{;} \KeywordTok{do}    \ExtensionTok{wgsim}\NormalTok{ Zea_mays.B73_RefGen_v4.dna.chromosome.Mt.fa -e 0 -d 500 -N 50000 -1 100 -2 100 -r 0.1  -R 0 -X 0 p1l}\VariableTok{$i}\NormalTok{.read1.fq l}\VariableTok{$i}\NormalTok{.read2.fq}\KeywordTok{;} \KeywordTok{done}
\FunctionTok{mv}\NormalTok{ l1.read1.fq p1l1.read1.fq }\CommentTok{#(Change name pop 1. all 10 idividuals)}
\end{Highlighting}
\end{Shaded}

\hypertarget{simulate-10-individals-population-2}{%
\subsection{simulate 10 individals population
2}\label{simulate-10-individals-population-2}}

--

\begin{Shaded}
\begin{Highlighting}[]
\KeywordTok{for} \ExtensionTok{i}\NormalTok{ in }\DataTypeTok{\{1..10\}}\KeywordTok{;} \KeywordTok{do}    \ExtensionTok{wgsim}\NormalTok{ Zea_mays.B73_RefGen_v4.dna.chromosome.Mt.fa -e 0 -d 500 -N 50000 -1 100 -2 100 -r 0.1  -R 0 -X 0 l}\VariableTok{$i}\NormalTok{.read1.fq l}\VariableTok{$i}\NormalTok{.read2.fq}\KeywordTok{;} \KeywordTok{done}
\FunctionTok{mv}\NormalTok{ l1.read1.fq p2l1.read1.fq }\CommentTok{#(Change name pop 2. all 10 idividuals)}
\end{Highlighting}
\end{Shaded}

\begin{longtable}[]{@{}l@{}}
\toprule
\endhead
\begin{minipage}[t]{0.04\columnwidth}\raggedright
\#\#\#\# check how many reads\strut
\end{minipage}\tabularnewline
\begin{minipage}[t]{0.04\columnwidth}\raggedright
\texttt{bash\ wc\ -l\ p1l1.read1.fq\ wc\ -l\ p2l1.read1.fq\ \#\ suppose\ to\ be\ 200,000\ lines\ =\ 50,000\ reads}\strut
\end{minipage}\tabularnewline
\bottomrule
\end{longtable}

\hypertarget{a-procedure-to-calculate-theta-and-f_st-values}{%
\section{\texorpdfstring{A procedure to calculate \(\theta\) and
\(F_{ST}\)
values}{A procedure to calculate \textbackslash{}theta and F\_\{ST\} values}}\label{a-procedure-to-calculate-theta-and-f_st-values}}

\hypertarget{align-the-ngs-reads-to-the-reference-genome}{%
\subsubsection{1. Align the NGS reads to the reference
genome}\label{align-the-ngs-reads-to-the-reference-genome}}

\begin{itemize}
\tightlist
\item
  \href{https://github.com/lh3/bwa}{bwa}
\item
  \href{https://github.com/samtools/samtools}{samtools}
\end{itemize}

\hypertarget{calculate-sfs}{%
\subsubsection{2. Calculate SFS}\label{calculate-sfs}}

\begin{itemize}
\tightlist
\item
  \href{http://www.popgen.dk/angsd/index.php/ANGSD}{ANGSD}
\end{itemize}

\hypertarget{calculate-the-thetas-and-fst-for-each-site}{%
\subsubsection{3. Calculate the thetas and Fst for each
site}\label{calculate-the-thetas-and-fst-for-each-site}}

\begin{itemize}
\tightlist
\item
  ANGSD
\end{itemize}

\hypertarget{submit-a-slurm-job-i-couldnt-do-it}{%
\section{Submit a Slurm job (I couldn't do
it)}\label{submit-a-slurm-job-i-couldnt-do-it}}

\begin{itemize}
\item
  We wrap our jobs in little batch scripts, which is nice because these
  also help make steps reproducible.
\item
  To keep your directory organized, I usually keep a scripts directory
  (i.e., \texttt{slurm-script/} ) and log dir (i.e., \texttt{slurm-log}
  ) for Slurm's logs.

  \begin{itemize}
  \tightlist
  \item
    Tip: use these logs, as these are very helpful in debugging. I
    separate them from my project because they fill up directories
    rather quickly.
  \end{itemize}
\item
  \texttt{D} sets your project directory.
\item
  \texttt{o} sets where standard output (of your batch script) goes.
\item
  \texttt{e} sets where standard error (of your batch script) goes.
\item
  \texttt{J} sets the job name.
\item
  \texttt{t} sets the time limit for the job, 24:00:00 indicates 24
  hours.
\item
  \texttt{-\/-mail}: will email you if the job is ``END'' or ``FAIL''
\end{itemize}

--

\begin{Shaded}
\begin{Highlighting}[]
\ExtensionTok{vi}\NormalTok{ my_theta.sh}

  \CommentTok{#!/bin/bash -l}
    \CommentTok{#SBATCH -D /home/agro932/sybarreral/Agro932hw}
    \CommentTok{#SBATCH -o /home/agro932/sybarreral/Agro932hw/slurm-log/stdout-%j.txt}
    \CommentTok{#SBATCH -e /home/agro932/sybarreral/Agro932hw/slurm-log/stderr-%j.txt}
    \CommentTok{#SBATCH -J theta}
    \CommentTok{#SBATCH -t 1:00:00}
    \CommentTok{#SBATCH --mail-user=sybarreral@huskers.unl.edu}
    \CommentTok{#SBATCH --mail-type=}\RegionMarkerTok{END}\CommentTok{ #email if ends}
    \CommentTok{#SBATCH --mail-type=FAIL #email if fails}
    \KeywordTok{set} \ExtensionTok{-e}
    \KeywordTok{set} \ExtensionTok{-u}

\CommentTok{# insert your script here}
\ExtensionTok{module}\NormalTok{ load bwa samtools}
\CommentTok{# alignment pop.1}
\BuiltInTok{cd}\NormalTok{ largedata/Thetas_Fst/bam_files}
\KeywordTok{for} \ExtensionTok{i}\NormalTok{ in }\DataTypeTok{\{1..10\}}\KeywordTok{;} \KeywordTok{do} \ExtensionTok{bwa}\NormalTok{ mem Zea_mays.B73_RefGen_v4.dna.chromosome.Mt.fa p1l}\VariableTok{$i}\NormalTok{.read1.fq p1l}\VariableTok{$i}\NormalTok{.read2.fq }\KeywordTok{|} \ExtensionTok{samtools}\NormalTok{ view -bSh - }\OperatorTok{>}\NormalTok{ p1l}\VariableTok{$i}\NormalTok{.bam}\KeywordTok{;} \KeywordTok{done}
\CommentTok{# alignment pop.2}
\KeywordTok{for} \ExtensionTok{i}\NormalTok{ in }\DataTypeTok{\{1..10\}}\KeywordTok{;} \KeywordTok{do} \ExtensionTok{bwa}\NormalTok{ mem Zea_mays.B73_RefGen_v4.dna.chromosome.Mt.fa p2l}\VariableTok{$i}\NormalTok{.read1.fq p2l}\VariableTok{$i}\NormalTok{.read2.fq }\KeywordTok{|} \ExtensionTok{samtools}\NormalTok{ view -bSh - }\OperatorTok{>}\NormalTok{ p2l}\VariableTok{$i}\NormalTok{.bam}\KeywordTok{;} \KeywordTok{done}
\CommentTok{# sort}
\KeywordTok{for} \ExtensionTok{i}\NormalTok{ in *.bam}\KeywordTok{;} \KeywordTok{do} \ExtensionTok{samtools}\NormalTok{ sort }\VariableTok{$i}\NormalTok{ -o sorted_}\VariableTok{$i}\KeywordTok{;} \KeywordTok{done}
\CommentTok{# index them}
\KeywordTok{for} \ExtensionTok{i}\NormalTok{ in sorted*.bam}\KeywordTok{;} \KeywordTok{do} \ExtensionTok{samtools}\NormalTok{ index }\VariableTok{$i}\KeywordTok{;} \KeywordTok{done}
\end{Highlighting}
\end{Shaded}

\hypertarget{a-procedure-to-calculate-theta-values}{%
\section{\texorpdfstring{A procedure to calculate \(\theta\)
values}{A procedure to calculate \textbackslash{}theta values}}\label{a-procedure-to-calculate-theta-values}}

\hypertarget{calculate-sfs-using-angsd}{%
\subsubsection{\texorpdfstring{2. Calculate SFS using
\texttt{ANGSD}}{2. Calculate SFS using ANGSD}}\label{calculate-sfs-using-angsd}}

\hypertarget{install-angsd-first}{%
\paragraph{Install ANGSD first}\label{install-angsd-first}}

\begin{Shaded}
\begin{Highlighting}[]
\BuiltInTok{cd}\NormalTok{ ~/bin/ }\CommentTok{# if you don't have one, do `mkdir bin`}
\FunctionTok{git}\NormalTok{ clone https://github.com/samtools/htslib.git}
\FunctionTok{git}\NormalTok{ clone https://github.com/ANGSD/angsd.git }
\BuiltInTok{cd}\NormalTok{ htslib}\KeywordTok{;} \FunctionTok{make}\KeywordTok{;} 
\BuiltInTok{cd}\NormalTok{ ../angsd}\KeywordTok{;}
\FunctionTok{make}\NormalTok{ HTSSRC=../htslib}
\end{Highlighting}
\end{Shaded}

\hypertarget{run-angsd}{%
\paragraph{run angsd}\label{run-angsd}}

\begin{Shaded}
\begin{Highlighting}[]
\CommentTok{#write the bam files to a txt file}
\FunctionTok{mkdir}\NormalTok{ bam_files}
\FunctionTok{mv}\NormalTok{ sorted*.bam bam_files}
\BuiltInTok{cd}\NormalTok{ bam_files/}
\FunctionTok{ls}\NormalTok{ sorted_p1*.bam }\OperatorTok{>}\NormalTok{ bam.txt}
\CommentTok{#see file 1}
\FunctionTok{cat}\NormalTok{ bam_p1.txt }
\FunctionTok{ls}\NormalTok{ sorted_p2*.bam }\OperatorTok{>}\NormalTok{ bam.txt}
\CommentTok{#see file 2}
\FunctionTok{cat}\NormalTok{ bam_p2.txt }
\end{Highlighting}
\end{Shaded}

\begin{center}\rule{0.5\linewidth}{0.5pt}\end{center}

\hypertarget{a-procedure-to-calculate-theta-values-1}{%
\section{\texorpdfstring{A procedure to calculate \(\theta\)
values}{A procedure to calculate \textbackslash{}theta values}}\label{a-procedure-to-calculate-theta-values-1}}

\begin{Shaded}
\begin{Highlighting}[]
\CommentTok{#calculate Sfs }
\ExtensionTok{../../../../angsd/angsd}\NormalTok{ -bam  bam.txt -doSaf 1 -anc ../Zea_mays.B73_RefGen_v4.dna.chromosome.Mt.fa -GL 1  -out out}
\CommentTok{# use realSFS to calculate sfs}
\ExtensionTok{../../../../angsd/misc/realSFS}\NormalTok{ out.saf.idx }\OperatorTok{>}\NormalTok{ out.sfs}
\end{Highlighting}
\end{Shaded}

\hypertarget{copy-the-result-to-cache-folder}{%
\paragraph{\texorpdfstring{Copy the result to \texttt{cache/}
folder}{Copy the result to cache/ folder}}\label{copy-the-result-to-cache-folder}}

\begin{Shaded}
\begin{Highlighting}[]
\CommentTok{## cp sfs to the cache/ folder}
\FunctionTok{cp}\NormalTok{ out.sfs ../../../cache/}
\end{Highlighting}
\end{Shaded}

\hypertarget{calculate-the-thetas-for-each-site}{%
\subsubsection{3. Calculate the thetas for each
site}\label{calculate-the-thetas-for-each-site}}

The output from the above command are two files out.thetas.gz and
out.thetas.idx. A formal description of these files can be found in the
doc/formats.pdf in the angsd package. It is possible to extract the
logscale persite thetas using the ./thetaStat print program.

\begin{Shaded}
\begin{Highlighting}[]
\ExtensionTok{../../../../angsd/angsd}\NormalTok{ -bam bam.txt -out out -doThetas 1 -doSaf 1 -pest out.sfs -anc ../Zea_mays.B73_RefGen_v4.dna.chromosome.Mt.fa -GL 1}
\ExtensionTok{../../../../angsd/misc/thetaStat}\NormalTok{ print out.thetas.idx }\OperatorTok{>}\NormalTok{ theta.txt}
\CommentTok{## cp theta to the cache/ folder}
\FunctionTok{cp}\NormalTok{ theta.txt ../../../cache/}
\end{Highlighting}
\end{Shaded}

\begin{center}\rule{0.5\linewidth}{0.5pt}\end{center}

\hypertarget{a-procedure-to-calculate-f_st}{%
\section{\texorpdfstring{A procedure to calculate
\(F_{ST}\)}{A procedure to calculate F\_\{ST\}}}\label{a-procedure-to-calculate-f_st}}

\hypertarget{create-two-list-bam-files}{%
\paragraph{Create two list bam files}\label{create-two-list-bam-files}}

\begin{Shaded}
\begin{Highlighting}[]
\FunctionTok{cp}\NormalTok{ bam_p1.txt pop1.txt}
\FunctionTok{cp}\NormalTok{ bam_p2.txt pop2.txt}
\end{Highlighting}
\end{Shaded}

\hypertarget{two-population-fst}{%
\paragraph{Two population Fst}\label{two-population-fst}}

\begin{Shaded}
\begin{Highlighting}[]
\CommentTok{# first calculate per pop saf for each populatoin}
\ExtensionTok{../../../../angsd/angsd}\NormalTok{ -b pop1.txt -anc ../Zea_mays.B73_RefGen_v4.dna.chromosome.Mt.fa -out pop1 -dosaf 1 -gl 1}
\ExtensionTok{../../../../angsd/angsd}\NormalTok{ -b pop2.txt -anc ../Zea_mays.B73_RefGen_v4.dna.chromosome.Mt.fa -out pop2 -dosaf 1 -gl 1}
\CommentTok{# calculate the 2dsfs prior}
\ExtensionTok{../../../../angsd/misc/realSFS}\NormalTok{ pop1.saf.idx pop2.saf.idx }\OperatorTok{>}\NormalTok{ pop1.pop2.ml}
\CommentTok{# prepare the fst for easy window analysis etc}
\ExtensionTok{../../../../angsd/misc/realSFS}\NormalTok{ fst index pop1.saf.idx pop2.saf.idx -sfs pop1.pop2.ml -fstout out}

\CommentTok{# get the global estimate}
\ExtensionTok{../../../../angsd/misc/realSFS}\NormalTok{ fst stats out.fst.idx}
\CommentTok{# below is not tested that much, but seems to work}
\ExtensionTok{../../../../angsd/misc/realSFS}\NormalTok{ fst stats2 out.fst.idx -win 500 -step 100 }\OperatorTok{>}\NormalTok{ fst_win.txt}
\end{Highlighting}
\end{Shaded}

\hypertarget{copy-the-result-to-cache-folder-1}{%
\paragraph{\texorpdfstring{Copy the result to \texttt{cache/}
folder}{Copy the result to cache/ folder}}\label{copy-the-result-to-cache-folder-1}}

\begin{Shaded}
\begin{Highlighting}[]
\CommentTok{## cp sfs to the cache/ folder}
\FunctionTok{cp}\NormalTok{ fst_win.txt ../../../cache/}
\end{Highlighting}
\end{Shaded}

\begin{center}\rule{0.5\linewidth}{0.5pt}\end{center}

\hypertarget{visualize-the-results}{%
\section{Visualize the results}\label{visualize-the-results}}

In local computer, using \texttt{R}:

\hypertarget{barplot-for-sfs}{%
\paragraph{Barplot for SFS}\label{barplot-for-sfs}}

\begin{Shaded}
\begin{Highlighting}[]
\CommentTok{#Barplot for SFS}
\NormalTok{sfs <-}\StringTok{ }\KeywordTok{scan}\NormalTok{(}\StringTok{'cache/out.sfs'}\NormalTok{)}
\NormalTok{sfs <-}\StringTok{ }\NormalTok{sfs[}\OperatorTok{-}\KeywordTok{c}\NormalTok{(}\DecValTok{1}\NormalTok{,}\KeywordTok{length}\NormalTok{(sfs))]}
\NormalTok{sfs <-}\StringTok{ }\NormalTok{sfs}\OperatorTok{/}\KeywordTok{sum}\NormalTok{(sfs)}

\KeywordTok{barplot}\NormalTok{(sfs,}\DataTypeTok{names=}\DecValTok{1}\OperatorTok{:}\KeywordTok{length}\NormalTok{(sfs), }\DataTypeTok{main=}\StringTok{'SFS'}\NormalTok{)}
\KeywordTok{barplot}\NormalTok{(sfs, }\DataTypeTok{col=}\StringTok{"#cdc0b0"}\NormalTok{, }\DataTypeTok{xlab=}\StringTok{"No. of segregating sites"}\NormalTok{, }
        \DataTypeTok{ylab=}\StringTok{"Minor allele frequency"}\NormalTok{, }
        \DataTypeTok{names=}\DecValTok{1}\OperatorTok{:}\KeywordTok{length}\NormalTok{(sfs))}
\end{Highlighting}
\end{Shaded}

\hypertarget{histgram-distribution-of-the-theta-values}{%
\paragraph{Histgram distribution of the theta
values}\label{histgram-distribution-of-the-theta-values}}

\begin{Shaded}
\begin{Highlighting}[]
\KeywordTok{library}\NormalTok{(}\StringTok{"data.table"}\NormalTok{)}
\CommentTok{#Histgram distribution of the theta values}
\CommentTok{#theta <- read.table("cache/theta.txt", header=TRUE)}
\NormalTok{theta <-}\StringTok{ }\KeywordTok{fread}\NormalTok{(}\StringTok{"cache/theta.txt"}\NormalTok{, }\DataTypeTok{data.table =}\OtherTok{FALSE}\NormalTok{)}
\KeywordTok{hist}\NormalTok{(theta}\OperatorTok{$}\NormalTok{Pairwise) }
\KeywordTok{hist}\NormalTok{(theta}\OperatorTok{$}\NormalTok{Pairwise, }\DataTypeTok{col=}\StringTok{"green"}\NormalTok{, }\DataTypeTok{xlab=}\StringTok{"Theta pairwise values (log10)"}\NormalTok{) }
\end{Highlighting}
\end{Shaded}

\hypertarget{scatter-plot-of-the-fst-values}{%
\paragraph{Scatter plot of the Fst
values}\label{scatter-plot-of-the-fst-values}}

\begin{Shaded}
\begin{Highlighting}[]
\CommentTok{#Scatter plot of the Fst values}
\NormalTok{fst <-}\StringTok{ }\KeywordTok{read.table}\NormalTok{(}\StringTok{"cache/fst_win.txt"}\NormalTok{, }\DataTypeTok{skip=}\DecValTok{1}\NormalTok{, }\DataTypeTok{header=}\OtherTok{FALSE}\NormalTok{)}
\KeywordTok{names}\NormalTok{(fst)[}\KeywordTok{c}\NormalTok{(}\DecValTok{3}\NormalTok{,}\DecValTok{5}\NormalTok{)] <-}\StringTok{ }\KeywordTok{c}\NormalTok{(}\StringTok{"midp"}\NormalTok{, }\StringTok{"fst"}\NormalTok{)}
\KeywordTok{plot}\NormalTok{(fst}\OperatorTok{$}\NormalTok{midp, fst}\OperatorTok{$}\NormalTok{fst, }\DataTypeTok{xlab=}\StringTok{"Physical position"}\NormalTok{, }\DataTypeTok{ylab=}\StringTok{"Fst"}\NormalTok{, }\DataTypeTok{col=}\StringTok{"#5f9ea0"}\NormalTok{, }\DataTypeTok{pch=}\DecValTok{16}\NormalTok{)}
\end{Highlighting}
\end{Shaded}

\begin{center}\rule{0.5\linewidth}{0.5pt}\end{center}

\hypertarget{general-feature-format-gff-from-ensemblplants}{%
\section{General feature format (GFF) from
EnsemblPlants}\label{general-feature-format-gff-from-ensemblplants}}

Maize \href{https://plants.ensembl.org/Zea_mays/Info/Index}{reference
genome}

change to \texttt{largedata\textbackslash{}lab4} folder:

\begin{Shaded}
\begin{Highlighting}[]
\FunctionTok{wget}\NormalTok{ ftp://ftp.ensemblgenomes.org/pub/plants/release-46/fasta/zea_mays/dna/Zea_mays.B73_RefGen_v4.dna.chromosome.Mt.fa.gz}

\CommentTok{### then unzip it}
\FunctionTok{gunzip}\NormalTok{ Zea_mays.B73_RefGen_v4.dna.chromosome.Mt.fa.gz}
\end{Highlighting}
\end{Shaded}

Similarly, we will download and unzip the
\href{ftp://ftp.ensemblgenomes.org/pub/plants/release-46/gff3/zea_mays/Zea_mays.B73_RefGen_v4.46.chromosome.Mt.gff3.gz}{Mt
GFF3} file

--

\hypertarget{use-r-to-process-the-gff3-file}{%
\paragraph{Use R to process the GFF3
file}\label{use-r-to-process-the-gff3-file}}

\begin{Shaded}
\begin{Highlighting}[]
\CommentTok{# install.package("data.table")}
\KeywordTok{library}\NormalTok{(}\StringTok{"data.table"}\NormalTok{)}
\CommentTok{## simply read in wouldn't work}
\NormalTok{gff <-}\StringTok{ }\KeywordTok{fread}\NormalTok{(}\StringTok{"largedata/Zea_mays.B73_RefGen_v4.46.chromosome.Mt.gff3"}\NormalTok{, }\DataTypeTok{skip=}\StringTok{"#"}\NormalTok{, }\DataTypeTok{header=}\OtherTok{FALSE}\NormalTok{, }\DataTypeTok{data.table=}\OtherTok{FALSE}\NormalTok{)}
\CommentTok{## grep -v means select lines that not matching any of the specified patterns}
\NormalTok{gff <-}\StringTok{ }\KeywordTok{fread}\NormalTok{(}\DataTypeTok{cmd=}\StringTok{'grep -v "#" largedata/Zea_mays.B73_RefGen_v4.46.chromosome.Mt.gff3'}\NormalTok{, }\DataTypeTok{header=}\OtherTok{FALSE}\NormalTok{, }\DataTypeTok{data.table=}\OtherTok{FALSE}\NormalTok{)}
\end{Highlighting}
\end{Shaded}

\begin{center}\rule{0.5\linewidth}{0.5pt}\end{center}

\hypertarget{general-feature-format-gff-version-3}{%
\section{General feature format (GFF) version
3}\label{general-feature-format-gff-version-3}}

\begin{verbatim}
  V1      V2         V3   V4     V5 V6 V7 V8
1 Mt Gramene chromosome    1 569630  .  .  .
2 Mt ensembl       gene 6391   6738  .  +  .
3 Mt    NCBI       mRNA 6391   6738  .  +  .
4 Mt    NCBI       exon 6391   6738  .  +  .
5 Mt    NCBI        CDS 6391   6738  .  +  0
6 Mt ensembl       gene 6951   8285  .  +  .
                  V9
1   ID=chromosome:Mt;Alias=AY506529.1,NC_007982.1;Is_circular=true
2   ID=gene:ZeamMp002;biotype=protein_coding;description=orf115-a1;
3   ID=transcript:ZeamMp002;Parent=gene:ZeamMp002;
4   Parent=transcript:ZeamMp002;Name=ZeamMp002.exon1;constitutive=1;ensembl_end_phase=0;
5   ID=CDS:ZeamMp002;Parent=transcript:ZeamMp002;
6   ID=gene:ZeamMp003;biotype=protein_coding;description=orf444
\end{verbatim}

\begin{center}\rule{0.5\linewidth}{0.5pt}\end{center}

\hypertarget{general-feature-format-gff-version-3-1}{%
\section{General feature format (GFF) version
3}\label{general-feature-format-gff-version-3-1}}

\begin{verbatim}
  V1      V2         V3   V4     V5 V6 V7 V8
1 Mt Gramene chromosome    1 569630  .  .  .
2 Mt ensembl       gene 6391   6738  .  +  .
                  V9
1   ID=chromosome:Mt;Alias=AY506529.1,NC_007982.1;Is_circular=true
2   ID=gene:ZeamMp002;biotype=protein_coding;description=orf115-a1;
\end{verbatim}

\begin{center}\rule{0.5\linewidth}{0.5pt}\end{center}

\begin{itemize}
\item
  1 \textbf{sequence}: The name of the sequence where the feature is
  located.
\item
  2 \textbf{source}: Keyword identifying the source of the feature, like
  a program (e.g.~RepeatMasker) or an organization (like ensembl).
\item
  3 \textbf{feature}: The feature type name, like ``gene'' or ``exon''.

  \begin{itemize}
  \tightlist
  \item
    In a well structured GFF file, all the children features always
    follow their parents in a single block (so all exons of a transcript
    are put after their parent ``transcript'' feature line and before
    any other parent transcript line).
  \end{itemize}
\item
  4 \textbf{start}: Genomic start of the feature, with a 1-base offset.

  \begin{itemize}
  \tightlist
  \item
    This is in contrast with other 0-offset half-open sequence formats,
    like \href{}{BED}.
  \end{itemize}
\item
  5 \textbf{end}: Genomic end of the feature, with a 1-base offset.

  \begin{itemize}
  \tightlist
  \item
    This is the same end coordinate as it is in 0-offset half-open
    sequence formats, like BED.
  \end{itemize}
\item
  6 \textbf{score}: Numeric value that generally indicates the
  confidence of the source in the annotated feature.

  \begin{itemize}
  \tightlist
  \item
    A value of ``.'' (a dot) is used to define a null value.
  \end{itemize}
\item
  7 \textbf{strand}: Single character that indicates the strand of the
  feature.

  \begin{itemize}
  \tightlist
  \item
    it can assume the values of ``+'' (positive, or 5' -\textgreater{}
    3'),
  \item
    ``-'', (negative, or 3' -\textgreater{} 5'), ``.'' (undetermined).
  \end{itemize}
\item
  8 \textbf{phase}: phase of CDS (\textbf{means CoDing Sequence})
  features.

  \begin{itemize}
  \tightlist
  \item
    The phase indicates where the feature begins with reference to the
    reading frame.
  \item
    The phase is one of the integers 0, 1, or 2, indicating the number
    of bases that should be removed from the beginning of this feature
    to reach the first base of the next codon.
  \end{itemize}
\item
  9 \textbf{attributes}: All the other information pertaining to this
  feature.

  \begin{itemize}
  \tightlist
  \item
    The format, structure and content of this field is the one which
    varies the most between the three competing file formats.
  \end{itemize}
\end{itemize}

\begin{center}\rule{0.5\linewidth}{0.5pt}\end{center}

\hypertarget{work-with-gff}{%
\section{Work with GFF}\label{work-with-gff}}

\begin{Shaded}
\begin{Highlighting}[]
\KeywordTok{names}\NormalTok{(gff) <-}\StringTok{ }\KeywordTok{c}\NormalTok{(}\StringTok{"seq"}\NormalTok{, }\StringTok{"source"}\NormalTok{, }\StringTok{"feature"}\NormalTok{, }\StringTok{"start"}\NormalTok{, }\StringTok{"end"}\NormalTok{, }\StringTok{"score"}\NormalTok{, }\StringTok{"strand"}\NormalTok{, }\StringTok{"phase"}\NormalTok{, }\StringTok{"att"}\NormalTok{)}
\KeywordTok{table}\NormalTok{(gff}\OperatorTok{$}\NormalTok{feature)}
\end{Highlighting}
\end{Shaded}

\hypertarget{get-genes-and-upstream-and-downstream-5kb-regions}{%
\subsubsection{Get genes and upstream and downstream 5kb
regions}\label{get-genes-and-upstream-and-downstream-5kb-regions}}

\begin{Shaded}
\begin{Highlighting}[]
\NormalTok{g <-}\StringTok{ }\KeywordTok{subset}\NormalTok{(gff, feature }\OperatorTok\StringTok{ "gene"}\NormalTok{)}
\NormalTok{g}\OperatorTok{$}\NormalTok{geneid <-}\StringTok{ }\KeywordTok{gsub}\NormalTok{(}\StringTok{".*gene:|;biotype.*"}\NormalTok{, }\StringTok{""}\NormalTok{, g}\OperatorTok{$}\NormalTok{att)}

\CommentTok{### + strand}
\NormalTok{gp <-}\StringTok{ }\KeywordTok{subset}\NormalTok{(g, strand }\OperatorTok\StringTok{ "+"}\NormalTok{) }
\CommentTok{# nrow(gp) 75}

\CommentTok{### get the 5k upstream of the + strand gene model}
\NormalTok{gp_up <-}\StringTok{ }\NormalTok{gp}
\NormalTok{gp_up}\OperatorTok{$}\NormalTok{end <-}\StringTok{ }\NormalTok{gp_up}\OperatorTok{$}\NormalTok{start }\OperatorTok{-}\StringTok{ }\DecValTok{1}
\NormalTok{gp_up}\OperatorTok{$}\NormalTok{start <-}\StringTok{ }\NormalTok{gp_up}\OperatorTok{$}\NormalTok{end }\OperatorTok{-}\StringTok{ }\DecValTok{5000} 

\CommentTok{### get the 5k downstream of the + strand gene model}
\NormalTok{gp_down <-}\StringTok{ }\NormalTok{gp}
\NormalTok{gp_down}\OperatorTok{$}\NormalTok{start <-}\StringTok{ }\NormalTok{gp_down}\OperatorTok{$}\NormalTok{end }\OperatorTok{+}\StringTok{ }\DecValTok{1}
\NormalTok{gp_down}\OperatorTok{$}\NormalTok{end <-}\StringTok{ }\NormalTok{gp_down}\OperatorTok{$}\NormalTok{start }\OperatorTok{+}\StringTok{ }\DecValTok{5000} 
\end{Highlighting}
\end{Shaded}

\begin{center}\rule{0.5\linewidth}{0.5pt}\end{center}

\hypertarget{get-genes-and-upstream-and-downstream-5kb-regions-1}{%
\subsubsection{Get genes and upstream and downstream 5kb
regions}\label{get-genes-and-upstream-and-downstream-5kb-regions-1}}

\begin{Shaded}
\begin{Highlighting}[]
\CommentTok{### - strand}
\NormalTok{gm <-}\StringTok{ }\KeywordTok{subset}\NormalTok{(g, strand }\OperatorTok\StringTok{ "-"}\NormalTok{) }
\KeywordTok{dim}\NormalTok{(gm) }\CommentTok{# 82}

\KeywordTok{fwrite}\NormalTok{(g, }\StringTok{"cache/mt_gene.txt"}\NormalTok{, }\DataTypeTok{sep=}\StringTok{"}\CharTok{\textbackslash{}t}\StringTok{"}\NormalTok{, }\DataTypeTok{row.names =} \OtherTok{FALSE}\NormalTok{, }\DataTypeTok{quote=}\OtherTok{FALSE}\NormalTok{)}
\end{Highlighting}
\end{Shaded}

\begin{center}\rule{0.5\linewidth}{0.5pt}\end{center}

\hypertarget{intepret-the-theta-results}{%
\subsection{Intepret the theta
results}\label{intepret-the-theta-results}}

\begin{Shaded}
\begin{Highlighting}[]
\KeywordTok{library}\NormalTok{(}\StringTok{"data.table"}\NormalTok{)}
\KeywordTok{library}\NormalTok{(}\StringTok{"GenomicRanges"}\NormalTok{)}
\KeywordTok{library}\NormalTok{(}\StringTok{"plyr"}\NormalTok{)}


\NormalTok{theta <-}\StringTok{ }\KeywordTok{fread}\NormalTok{(}\StringTok{"cache/theta.txt"}\NormalTok{, }\DataTypeTok{data.table=}\OtherTok{FALSE}\NormalTok{)}
\KeywordTok{names}\NormalTok{(theta)[}\DecValTok{1}\NormalTok{] <-}\StringTok{ "seq"}

\NormalTok{up5k <-}\StringTok{ }\KeywordTok{read.table}\NormalTok{(}\StringTok{"cache/mt_gene_up5k.txt"}\NormalTok{, }\DataTypeTok{header=}\OtherTok{TRUE}\NormalTok{)}

\CommentTok{### define the subject file for theta values}
\NormalTok{grc <-}\StringTok{ }\KeywordTok{with}\NormalTok{(theta, }\KeywordTok{GRanges}\NormalTok{(}\DataTypeTok{seqnames=}\NormalTok{seq, }\KeywordTok{IRanges}\NormalTok{(}\DataTypeTok{start=}\NormalTok{Pos, }\DataTypeTok{end=}\NormalTok{Pos)))}

\CommentTok{### define the query file for genomic feature}
\NormalTok{grf <-}\StringTok{ }\KeywordTok{with}\NormalTok{(up5k, }\KeywordTok{GRanges}\NormalTok{(}\DataTypeTok{seqnames=}\NormalTok{seq, }\KeywordTok{IRanges}\NormalTok{(}\DataTypeTok{start=}\NormalTok{start, }\DataTypeTok{end=}\NormalTok{end), }\DataTypeTok{geneid=}\NormalTok{geneid))}
    
\CommentTok{### find overlaps between the two}
\NormalTok{tb <-}\StringTok{ }\KeywordTok{findOverlaps}\NormalTok{(}\DataTypeTok{query=}\NormalTok{grf, }\DataTypeTok{subject=}\NormalTok{grc)}
\NormalTok{tb <-}\StringTok{ }\KeywordTok{as.matrix}\NormalTok{(tb)}
    
\NormalTok{out1 <-}\StringTok{ }\KeywordTok{as.data.frame}\NormalTok{(grf[tb[,}\DecValTok{1}\NormalTok{]])}
\NormalTok{out2 <-}\StringTok{ }\KeywordTok{as.data.frame}\NormalTok{(grc[tb[,}\DecValTok{2}\NormalTok{]])}
\CommentTok{### for each genomic feature, find the sites with non-missing data}
\NormalTok{out <-}\StringTok{ }\KeywordTok{cbind}\NormalTok{(out1, out2[, }\StringTok{"start"}\NormalTok{]) }
\KeywordTok{names}\NormalTok{(out)[}\KeywordTok{ncol}\NormalTok{(out)] <-}\StringTok{ "pos"}
\end{Highlighting}
\end{Shaded}

\begin{center}\rule{0.5\linewidth}{0.5pt}\end{center}

\hypertarget{intepret-the-theta-results-1}{%
\subsection{Intepret the theta
results}\label{intepret-the-theta-results-1}}

\begin{Shaded}
\begin{Highlighting}[]
\CommentTok{#define unique identifier and merge with the thetas}
\NormalTok{out}\OperatorTok{$}\NormalTok{uid <-}\StringTok{ }\KeywordTok{paste}\NormalTok{(out}\OperatorTok{$}\NormalTok{seqnames, out}\OperatorTok{$}\NormalTok{pos, }\DataTypeTok{sep=}\StringTok{"_"}\NormalTok{)}
\NormalTok{theta}\OperatorTok{$}\NormalTok{uid <-}\StringTok{ }\KeywordTok{paste}\NormalTok{(theta}\OperatorTok{$}\NormalTok{seq, theta}\OperatorTok{$}\NormalTok{Pos, }\DataTypeTok{sep=}\StringTok{"_"}\NormalTok{)}

\NormalTok{df <-}\StringTok{ }\KeywordTok{merge}\NormalTok{(out, theta[, }\KeywordTok{c}\NormalTok{(}\OperatorTok{-}\DecValTok{1}\NormalTok{, }\DecValTok{-2}\NormalTok{)], }\DataTypeTok{by=}\StringTok{"uid"}\NormalTok{)}
\CommentTok{# for each upstream 5k region, how many theta values}

\NormalTok{mx <-}\StringTok{ }\KeywordTok{ddply}\NormalTok{(df, .(geneid), summarise,}
            \DataTypeTok{Pairwise =} \KeywordTok{mean}\NormalTok{(Pairwise, }\DataTypeTok{na.rm=}\OtherTok{TRUE}\NormalTok{),}
            \DataTypeTok{thetaH =} \KeywordTok{mean}\NormalTok{(thetaH, }\DataTypeTok{na.rm=}\OtherTok{TRUE}\NormalTok{),}
            \DataTypeTok{nsites =} \KeywordTok{length}\NormalTok{(uid))}
\end{Highlighting}
\end{Shaded}

\begin{center}\rule{0.5\linewidth}{0.5pt}\end{center}

\hypertarget{intepret-the-theta-results-2}{%
\subsection{Intepret the theta
results}\label{intepret-the-theta-results-2}}

Copy and paste everything above, and pack it into an \texttt{R}
function:

\begin{Shaded}
\begin{Highlighting}[]
\NormalTok{get_mean_theta <-}\StringTok{ }\ControlFlowTok{function}\NormalTok{(}\DataTypeTok{gf_file=}\StringTok{"cache/mt_gene_up5k.txt"}\NormalTok{)\{}
  \CommentTok{# gf_file: gene feature file [chr, ="cache/mt_gene_up5k.txt"]}
  
\NormalTok{  theta <-}\StringTok{ }\KeywordTok{fread}\NormalTok{(}\StringTok{"cache/theta.txt"}\NormalTok{, }\DataTypeTok{data.table=}\OtherTok{FALSE}\NormalTok{)}
  \KeywordTok{names}\NormalTok{(theta)[}\DecValTok{1}\NormalTok{] <-}\StringTok{ "seq"}

\NormalTok{  up5k <-}\StringTok{ }\KeywordTok{read.table}\NormalTok{(gf_file, }\DataTypeTok{header=}\OtherTok{TRUE}\NormalTok{)}

  \CommentTok{### define the subject file for theta values}
\NormalTok{  grc <-}\StringTok{ }\KeywordTok{with}\NormalTok{(theta, }\KeywordTok{GRanges}\NormalTok{(}\DataTypeTok{seqnames=}\NormalTok{seq, }\KeywordTok{IRanges}\NormalTok{(}\DataTypeTok{start=}\NormalTok{Pos, }\DataTypeTok{end=}\NormalTok{Pos)))}

  \CommentTok{### define the query file for genomic feature}
\NormalTok{  grf <-}\StringTok{ }\KeywordTok{with}\NormalTok{(up5k, }\KeywordTok{GRanges}\NormalTok{(}\DataTypeTok{seqnames=}\NormalTok{seq, }\KeywordTok{IRanges}\NormalTok{(}\DataTypeTok{start=}\NormalTok{start, }\DataTypeTok{end=}\NormalTok{end), }\DataTypeTok{geneid=}\NormalTok{geneid))}
    
  \CommentTok{### find overlaps between the two}
\NormalTok{  tb <-}\StringTok{ }\KeywordTok{findOverlaps}\NormalTok{(}\DataTypeTok{query=}\NormalTok{grf, }\DataTypeTok{subject=}\NormalTok{grc)}
\NormalTok{  tb <-}\StringTok{ }\KeywordTok{as.matrix}\NormalTok{(tb)}
    
\NormalTok{  out1 <-}\StringTok{ }\KeywordTok{as.data.frame}\NormalTok{(grf[tb[,}\DecValTok{1}\NormalTok{]])}
\NormalTok{  out2 <-}\StringTok{ }\KeywordTok{as.data.frame}\NormalTok{(grc[tb[,}\DecValTok{2}\NormalTok{]])}
  \CommentTok{### for each genomic feature, find the sites with non-missing data}
\NormalTok{  out <-}\StringTok{ }\KeywordTok{cbind}\NormalTok{(out1, out2[, }\StringTok{"start"}\NormalTok{]) }
  \KeywordTok{names}\NormalTok{(out)[}\KeywordTok{ncol}\NormalTok{(out)] <-}\StringTok{ "pos"}

  \CommentTok{#define unique identifier and merge with the thetas}
\NormalTok{  out}\OperatorTok{$}\NormalTok{uid <-}\StringTok{ }\KeywordTok{paste}\NormalTok{(out}\OperatorTok{$}\NormalTok{seqnames, out}\OperatorTok{$}\NormalTok{pos, }\DataTypeTok{sep=}\StringTok{"_"}\NormalTok{)}
\NormalTok{  theta}\OperatorTok{$}\NormalTok{uid <-}\StringTok{ }\KeywordTok{paste}\NormalTok{(theta}\OperatorTok{$}\NormalTok{seq, theta}\OperatorTok{$}\NormalTok{Pos, }\DataTypeTok{sep=}\StringTok{"_"}\NormalTok{)}

\NormalTok{  df <-}\StringTok{ }\KeywordTok{merge}\NormalTok{(out, theta[, }\KeywordTok{c}\NormalTok{(}\OperatorTok{-}\DecValTok{1}\NormalTok{, }\DecValTok{-2}\NormalTok{)], }\DataTypeTok{by=}\StringTok{"uid"}\NormalTok{)}
  \CommentTok{# for each upstream 5k region, how many theta values}

\NormalTok{  mx <-}\StringTok{ }\KeywordTok{ddply}\NormalTok{(df, .(geneid), summarise,}
            \DataTypeTok{Pairwise =} \KeywordTok{mean}\NormalTok{(Pairwise, }\DataTypeTok{na.rm=}\OtherTok{TRUE}\NormalTok{),}
            \DataTypeTok{thetaH =} \KeywordTok{mean}\NormalTok{(thetaH, }\DataTypeTok{na.rm=}\OtherTok{TRUE}\NormalTok{),}
            \DataTypeTok{nsites =} \KeywordTok{length}\NormalTok{(uid))}
  \KeywordTok{return}\NormalTok{(mx)}
\NormalTok{\}}
\end{Highlighting}
\end{Shaded}

\begin{center}\rule{0.5\linewidth}{0.5pt}\end{center}

\hypertarget{plot-the-results}{%
\subsection{Plot the results}\label{plot-the-results}}

Run the customized R function

\begin{Shaded}
\begin{Highlighting}[]
\CommentTok{### apply the function}
\NormalTok{up5k <-}\StringTok{ }\KeywordTok{get_mean_theta}\NormalTok{(}\DataTypeTok{gf_file=}\StringTok{"cache/mt_gene_up5k.txt"}\NormalTok{)}
\NormalTok{down5k <-}\StringTok{ }\KeywordTok{get_mean_theta}\NormalTok{(}\DataTypeTok{gf_file=}\StringTok{"cache/mt_gene_down5k.txt"}\NormalTok{)}
\NormalTok{gene <-}\StringTok{ }\KeywordTok{get_mean_theta}\NormalTok{(}\DataTypeTok{gf_file=}\StringTok{"cache/mt_gene.txt"}\NormalTok{)}
\end{Highlighting}
\end{Shaded}

And then plot the results:

\begin{Shaded}
\begin{Highlighting}[]
\KeywordTok{library}\NormalTok{(}\StringTok{"ggplot2"}\NormalTok{)}

\NormalTok{up5k}\OperatorTok{$}\NormalTok{feature <-}\StringTok{ "up 5k"}
\NormalTok{down5k}\OperatorTok{$}\NormalTok{feature <-}\StringTok{ "down 5k"}
\NormalTok{gene}\OperatorTok{$}\NormalTok{feature <-}\StringTok{ "genic"}
\NormalTok{res <-}\StringTok{ }\KeywordTok{rbind}\NormalTok{(up5k, down5k)}
\NormalTok{res_nongenic <-}\StringTok{ }\NormalTok{res}
\NormalTok{res_nongenic}\OperatorTok{$}\NormalTok{feature <-}\StringTok{ "intergenic"}
\NormalTok{res_t <-}\StringTok{ }\KeywordTok{rbind}\NormalTok{(res, gene, res_nongenic)}
\KeywordTok{ggplot}\NormalTok{(res_t, }\KeywordTok{aes}\NormalTok{(}\DataTypeTok{x=}\NormalTok{feature, }\DataTypeTok{y=}\NormalTok{Pairwise, }\DataTypeTok{fill=}\NormalTok{feature)) }\OperatorTok{+}\StringTok{ }
\StringTok{  }\KeywordTok{geom_violin}\NormalTok{(}\DataTypeTok{trim=}\OtherTok{FALSE}\NormalTok{)}\OperatorTok{+}
\StringTok{  }\KeywordTok{labs}\NormalTok{(}\DataTypeTok{title=}\StringTok{"Theta value"}\NormalTok{, }\DataTypeTok{x=}\StringTok{""}\NormalTok{, }\DataTypeTok{y =} \StringTok{"Log10 (theta)"}\NormalTok{)}\OperatorTok{+}
\StringTok{  }\KeywordTok{geom_boxplot}\NormalTok{(}\DataTypeTok{width=}\FloatTok{0.1}\NormalTok{, }\DataTypeTok{fill=}\StringTok{"white"}\NormalTok{)}\OperatorTok{+}
\StringTok{  }\KeywordTok{scale_fill_brewer}\NormalTok{(}\DataTypeTok{palette=}\StringTok{"Blues"}\NormalTok{) }\OperatorTok{+}\StringTok{ }
\StringTok{  }\KeywordTok{theme_classic}\NormalTok{()}
\end{Highlighting}
\end{Shaded}

\begin{center}\rule{0.5\linewidth}{0.5pt}\end{center}

\end{document}
